% For revision control
\rcsid{$Id$}
\rcsid{$Header$}
\rcskwsave{$Author$}
\rcskwsave{$Date$} 
\rcskwsave{$Revision$}
\rcsRegisterAuthor{devangel}{Dennis J. Evangelista}



\chapter{Solids}
\label{ch:Solids}
\index{solids}
% chapter author here?
% chapter abstract here?
Solid materials are ubiquitous in biology. Animals have hard claws that they use to dig or climb with. Mollusks have shells to protect their soft bodies. Your bones give your muscles something stiff to move around. If things weren't solid, nothing would have much structure and we'd all be living in a primordial soup.

You probably think you know what a solid is. But what makes a solid so solid?Our first intuition is that a solid can be picked up and put down. It's got a shape that doesn't change with time. These intuitive feelings you have about solids can be translated into equations. In fact, Richard Hooke did so in the 1600's. He came up with these general features of ideal solids: 

\begin{itemize}
\item Solids have a shape
\item Solids can be stretched or squished
\item Solids resist shear
\item Under enough force, solids can break
\end{itemize}

Let play with a block of a solid, and for this example it's a block of rubber. If we stretch it, it will resist. How much? That depends on how hard you pull and the cross-sectional area of the block. This is called stress, and is represented with $\sigma$.
\begin{equation}
\sigma = \frac{F}{A}
\end{equation}
where F is the force, and A is the cross-sectional area of the block.

Say we pull pretty hard on the block of rubber. It's probably going to stretch. If you have a huge block and a tiny block and you stretch them to the same length, one is going to resist much more than the other. This is why we have to scale the stretch to the length of the original block. This is called strain, and is represented with $\varepsilon$.
\begin{equation}
\varepsilon = \frac{L-L_o}{L_o}
\end{equation}
where $L_o$ is the resting, or original length of the solid, and $L$ is the stretched length of the solid.

Now that we have ways to measure the behavior of solids, we'd like to be able to predict the behavior of solids. If we know how hard we're pulling on a solid, we'd like to be able to know how much it has changed in length, and vice versa. We do this with the Young's Modulus, which is represented with $E$.
\begin{equation}
E = \frac{\sigma}{\varepsilon}
\end{equation}

If the Young's Modulus is independent of stress and strain, we call this solid an Ideal or Hookean Solid.

Poisson's ratio

Solids are evil. That is why the Dominion \index{Dominion War} wages war against them \citep{Son-of-Moog:2488}.

\section{History of solid aggression in the Gamma Quadrant}
\section{Weaknesses of solids and Dominion anti-solid strategies and tactics}
\begin{figure}[h!]
 \label{jaculus}
 \centering
  \includegraphics[width=0.5\textwidth]{ch-solids/figures/jaculus.jpg}
 \caption{The Dipodid \species{Jaculus jaculus}\index{Jaculus jaculus@\textit{Jaculus jaculus}}, believed to resemble the ancestor of crown-group Voortaidae.}
\end{figure}


\bibliographystyle{apalike}
\bibliography{references/ch-solids}
