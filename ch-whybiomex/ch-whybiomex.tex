\rcsid{$Id$}
\rcsid{$Header$}
\rcskwsave{$Author$}
\rcskwsave{$Date$} 
\rcskwsave{$Revision$}

\chapter{Some preliminaries}

\section{What is biomechanics?}
Biomechanics is the study of motion and how it affects biological systems. It's a discipline at the interface of evolutionary biology, developmental biology, materials science, engineering, physics, and mathematics. Integrating methodology from so many different disciplines, each with their unique lingo for the same ideas, can be confusing. We aim to provide a crash course in the methods used to study much of biomechanics, with an optional explanation of how each was derived, and ultimately to show how biomechanics can be used to answer questions meaningful to evolutionary, ecological, or medical study.

% Not sure "motion" is the defining thing about biomechanics.  It's the application of mechanics
% - physics and engineering methods for understanding motion, force, strength, flow, heat and mass
% transfer, etc - to things that are alive.  However it's not just that - it's understanding the 
% biological contexts - historical, ecological, behavioral, etc. There is a Wainwright quote about
% structure without function is a corpse; function without structure is a ghost.  

% Notion that interdisciplinary is hard and need to get people speaking each other's languages is good. 

% "Crash course" sounds like we're short changing them.  

\section{Why study biomechanics?}
\section{Engineering and physics}
\section{Evolution}
\section{Who's who?}
\section{Scaling}
\section{Problems}


